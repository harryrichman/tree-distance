%v1
\documentclass{amsart}
\usepackage{amssymb, amsmath, amsfonts, amsthm}
\usepackage{graphics, enumerate, mathrsfs, mathtools,tikz-cd,soul,csquotes,dsfont}
\usetikzlibrary{positioning,arrows,scopes}
%\usepackage[left=1.25in,
%			right=1.25in,
%			top=1.25in,
%			bottom=1.25in]{geometry}
%\usepackage{fancyhdr}
%\pagestyle{fancy}

% COMMENT OUT FOR FINAL VERSION
\usepackage{showkeys}

\newtheorem{thm}{Theorem}
\newtheorem*{thm*}{Theorem}
\newtheorem{obs}[thm]{Observation}
\newtheorem{prop}[thm]{Proposition}
\newtheorem{lem}[thm]{Lemma}
\newtheorem{cor}[thm]{Corollary}

\theoremstyle{definition}
\newtheorem{prob}[thm]{Problem}
\newtheorem{dfn}[thm]{Definition}
\newtheorem{eg}[thm]{Example}
\newtheorem{rmk}[thm]{Remark}
\newtheorem{conj}[thm]{Conjecture}

\newcommand{\CC}{\mathbb{C}}
\newcommand{\FF}{\mathbb{F}}
\newcommand{\RR}{\mathbb{R}}
\newcommand{\ZZ}{\mathbb{Z}}
\newcommand{\QQ}{\mathbb{Q}}
\newcommand{\NN}{\mathbb{N}}
\newcommand{\PP}{\mathbb{P}}
% indicator function
\newcommand{\one}{\mathds{1}}
\newcommand{\cO}{\mathcal{O}}
\newcommand{\cU}{\mathcal{U}}
\newcommand{\cL}{\mathcal{L}}
\newcommand{\cC}{\mathcal{C}}
\newcommand{\cE}{\mathcal{E}}

% sum of cofactors
\DeclareMathOperator{\cof}{cof}
% convex hull of vertex set
\DeclareMathOperator{\conv}{conv}
\DeclareMathOperator{\energy}{\cE}
\DeclareMathOperator{\supp}{supp}
%%%% capacity
\DeclareMathOperator{\Capacity}{\textsc{cap}}
\newcommand{\posCap}{c_{pos}}
\DeclareMathOperator{\argmax}{argmax}
\DeclareMathOperator{\argmin}{argmin}
\DeclareMathOperator{\val}{val}
%topological interior
\DeclareMathOperator{\interior}{int}

\DeclareMathOperator{\rspan}{row.span}
\DeclareMathOperator{\Sym}{Sym}
\DeclareMathOperator{\indeg}{indeg}
%\DeclareMathOperator{\ker}{ker}
\DeclareMathOperator{\coker}{coker}
\DeclareMathOperator{\PL}{PL}
%%% use either \Delta or Div %%%
\DeclareMathOperator{\zspan}{span}
\DeclareMathOperator{\im}{im}
\DeclareMathOperator{\Br}{Brk}
\DeclareMathOperator{\ev}{ev}
\DeclareMathOperator{\vol}{vol}
\DeclareMathOperator{\coeff}{coeff}

% spanning trees
\newcommand{\trees}{\mathcal{T}}
\newcommand{\forests}{\mathcal{F}}
\renewcommand{\Re}{\mathrm{Re}}
\newcommand{\dsp}{\displaystyle}
\newcommand{\pderiv}[1]{\frac{\partial}{\partial{#1}}}
\newcommand{\angles}[1]{\langle {#1} \rangle}

% for comments
\newcommand{\note}[1]{{\color{red} \sf $\diamondsuit$  {#1} $\diamondsuit$ }}
\newcommand{\todo}[1]{\note{#1}}
\newcommand{\harry}[1]{{\color{red} \sf $\diamondsuit$  Harry: {#1} $\diamondsuit$ }}

\begin{document}
\title[Tree distance minors]{Minors of tree distance matrices}
\author{Harry Richman}
\author{Farbod Shokrieh}
\author{Chenxi Wu}
\date{v1, \today  \,(Preliminary draft, not for circulation).}
%\thanks{This work was partially supported by ....}


\begin{abstract}
We prove a formula for the determinant of 
a principal minor of
the distance matrix of a weighted tree.
This generalizes a result of Graham and Pollak.
\end{abstract}
\maketitle

\setcounter{tocdepth}{1}
\tableofcontents

\section{Introduction}

Suppose $G = (V,E)$ is a tree with $n$ vertices.
Let $D$ denote the distance matrix of $G$.
In
\cite{graham-pollak},
Graham and Pollak proved that
\begin{equation}\label{eq:full-det}
\det D = (-1)^{n-1} 2^{n-2} (n-1). 
\end{equation}
This identity is remarkable in that the result does not depend on the tree structure,
beyond the number of vertices.
This identity was motivated by a problem in data communication,
and inspired much further research on distance matrices.
%Self-contained proofs are also given in \cite{du-yeh,yan-yeh}.


The main result of this paper is to generalize \eqref{eq:full-det} by replacing $D$ with any principal submatrix.
For $S \subset V(G)$, let $D[S]$ denote the principal minor consisting of the $S$-indexed rows and columns.
\begin{thm}
\label{thm:main}
Suppose $G$ is a tree with $n$ vertices, 
and distance matrix $D$.
Let $S \subset V(G)$ be a subset of vertices.
Then
\begin{equation}\label{eq:main}
\det D[S] = (-1)^{|S|-1} 2^{|S|-2} \left( (n-1)\, \kappa(G/S)  - \sum_{\mathcal F_2(G/S)} k(F_{*})^2  \right).
\end{equation}
where 
$G/S$ denotes the quotient graph that identifies together vertices in $S$,
$\kappa(G/S)$ is the number of $S$-rooted spanning forests of $G$,
$\forests_2(G/S)$ is the set of $(S,*)$-rooted spanning forests of $G$,
$F_{*}$ denotes the $*$-component of $F$, 
and
\begin{equation*}
k(F_{*}) 
%= \sum_{x \in V( F_{*})} {2 - \deg(x)} 
= 2 - \deg^o(F,*).
\end{equation*}
\end{thm}
Note: the quantities $c(F,*)$, $k(F_*)$ satisfy
$$
1 \leq c(F,*) \leq |S|, \qquad
2 - |S| \leq k(F_*) \leq 1 .
$$
When $S = V$ is the full vertex set, the quotient graph $G / V$ consists of a single vertex with $n-1$ loop edges, 
so $\kappa(G/V) = 1$ and $\forests_2(G/V) = \emptyset$. 


Weighted version:
A weighted version of \eqref{eq:full-det} was proved by Bapat--Kirkland--Neumann~\cite{bapat-kirkland-neumann}.
\begin{equation}\label{eq:w-full-det}
\det D_{\alpha} = (-1)^{n-1} 2^{n-2} \prod_{e \in E} \alpha_e \sum_{e \in E} \alpha_e .
\end{equation}
\begin{thm}
\label{thm:w-main}
Suppose $G$ is a finite, weighted tree, and $A \subset V(G)$ is a subset of vertices.
Then
\begin{equation}\label{eq:w-main}
\det D[A] = (-1)^{|S|-1} 2^{|S|-2} \left( \sum_{E(G)}\ell(e) \sum_{\trees(G/S)} w(T) - \sum_{\forests_2(G/S)} w(F) k(F_{*})^2 \right).
\end{equation}
where 
$\mathcal T(G/A)$ denotes the set of $A$-rooted spanning forests of $G$,
$F_2$ varies over all $(A,*)$-rooted spanning forests of $G$,
$F_{*}$ denotes the $*$-component of $F$.
\end{thm}

It is worth observing that the distances appearing in $D[S]$ may ignore a large part of the ambient tree $G$.
We could instead replace $G$ by the subtree consisting of paths between vertices in $S$,
which we call $\conv(S,G)$, the {\em convex hull} of $S \subset G$.
To apply formula \eqref{eq:main} ``efficiently'', 
we should replace $G$ with this convex hull $\conv(S,G)$.
However, the formula as stated is true even without this replacement due to cancellation of terms.

\begin{cor}
\begin{equation}
\frac{\det D[S]}{\cof D[S]} = \frac12 \left( \sum_{E(G)} \ell(e) - \frac{\sum_{\forests_2(G/S)} w(F) k(F_*)^2}{\sum_{\trees(G/S)} w(T)} \right) .
\end{equation}
\end{cor}

\begin{thm}[Monotonicity of principal minor ratios]
Suppose $G = (V,E)$ is a finite, weighted tree with distance matrix $D$.
\begin{enumerate}
\item 
If $S \subset V(G)$ is nonempty,
\begin{equation*}
0 \leq \frac{\det D[S]}{\cof D[S]} \leq \frac12 \sum_{E(G)} \ell(e) .
\end{equation*}

\item 
If $\conv(S,G)$ denotes the subtree of $G$ consisting of all paths between points of $S \subset V(G)$,
\begin{equation*}
 \frac{\det D[S]}{\cof D[S]} \leq \frac12 \sum_{E(\conv(S, G))} \ell(e) .
\end{equation*}

\item 
If $A,B \subset V(G)$
are nonempty subsets with
$A \subset B$,
then
\begin{equation*}
 \frac{\det D[A]}{\cof D[A]}  \leq \frac{\det D[B]}{\cof D[B]}  .
\end{equation*}
\end{enumerate}
\end{thm}

\begin{thm}[Nonsingular minors]
Let $G$ be a finite, weighted tree
with distance matrix $D$,
and let $S \subset V(G)$ be a subset of vertices.
If $|S|\geq 2$ then $\det D[S] \neq 0$.
\end{thm}

\subsection{Previous work} 
A formula for the inverse matrix $D^{-1}$ was found by Graham and Lov\'{a}sz in \cite{graham-lovasz}.

\subsection{Notation}

% $\Gamma$ a compact metric graph

$G$ a finite graph, 
loops and parallel edges allowed,
possibly disconnected

$E(G)$ edge set of $G$

$V(G)$ vertex set of $G$

%$\ell : E(G) \to \RR_{>0}$
%is a length function on edges of $G$

% $\cC(\Gamma)$ continuous $\RR$-valued functions on $\Gamma$

% $Meas(\Gamma)$ signed Borel measures on $\Gamma$

% $Meas_{\geq 0}(\Gamma)$ positive Borel measures on $\Gamma$

\section{Background}

For background on enumeration problems for graphs and trees, see Moon \cite{moon}.



The following theorem is due to Kirchhoff.
For any graph $G$, let $\kappa(G)$ denote the number of spanning trees of $G$.
\begin{thm}[All-minors matrix tree theorem]
Let $G = (V,E)$ be a finite graph.
Let $L$ denote the Laplacian matrix of $G$.
Then for any nonempty vertex set $S \subset V(G)$,
\begin{equation}
\det L[V \setminus S] = \kappa( G / S) .
\end{equation}
\end{thm}
Note that $\kappa(G/S)$ is also the number of $S$-rooted spanning forests of $G$.


\begin{thm}[\cite{bapat-sivasubramanian}]
Let $T$ be a tree with $m+1$ vertices and $m$ edges.
Let $D$ be the distance matrix of $T$, and $L$ the Laplacian matrix.
Let $S \subset V(T)$ be a subset of vertices of $T$. 
Then
\begin{equation*}
\cof D[S] = (-2)^{|S|-1} \det L[V \setminus S] .
\end{equation*}
\end{thm}

\subsection{Trees and forests}


\section{Proofs}

Outline of proof: given subset $S$ and distance matrix minor $D[S]$, we will
\begin{enumerate}
\item 
Find vector $\mathbf{m} \in \RR^S$ such that $D[S]\mathbf{m} = \lambda \mathbf{1}$.

\item 
Compute the sum of entries of $\mathbf{m}$, i.e. $\mathbf{1}^T\mathbf{m}$.

\item 
Note the identity
$$ \mathbf{1}^T \mathbf{m} 
= \lambda (\mathbf{1}^T D[S]^{-1} \mathbf{1}) 
= \lambda \frac{\cof D[S]}{\det D[S]} .$$
where $\cof D[S]$ is the sum of cofactors of $D[S]$.

\item 
Use known expression for $\cof D[S]$ to compute
$$
\det D[S] = \lambda (\cof D[S]) \left( \mathbf{1}^T \mathbf{m} \right)^{-1}.
$$
\end{enumerate}
The interesting part  of this expression will be located in the constant $\lambda$.

\begin{eg}
Suppose $G$ is a tree consisting of three paths joined at a central vertex.
Let $S$ consist of the central vertex, and the three endpoints of the paths. 
The corresponding minor of the distance matrix is
$$
D[S] = \begin{bmatrix}
0 & a & b & c \\
a & 0 & a + b & a + c \\
b & a + b & 0 & b + c \\
c & a + c & b + c & 0
\end{bmatrix}.
$$
Following the steps outlined above:
\begin{enumerate}
\item 
The vector $\mathbf{m} = \begin{bmatrix} -1 \\ 1 \\ 1 \\ 1 \end{bmatrix}$ satisfies
$
D[S] \mathbf{m} = (a+b+c) \mathbf{1}
$

\item 
The sum of entries of $\mathbf{m}$ is $\mathbf{1}^T \mathbf{m} = 2$.

\item 
We have 
$$2 = \mathbf{1}^T \mathbf{m} = \lambda (\mathbf{1}^T D[S] \mathbf{1}) = \displaystyle \lambda \frac{\cof D[S]}{\det D[S]}.$$

\item 
The cofactor sum is 
$\cof D[S] = -8 abc$,
so the determinant is
\begin{align*}
\det D[S] 
= \lambda \frac{\cof A}{\mathbf{1}^T \mathbf{m}}
= (a+b+c) (-8 abc)\frac{1}{2}
= -4(a+b+c)abc.
\end{align*}
\end{enumerate}
\end{eg}

\begin{prop}
Let $T = (V,E)$ a tree, and consider the vector $\mathbf{m} \in \RR^V$
defined by 
\begin{equation*}
\mathbf{m}(v) = 2 - \deg v,
\end{equation*}
where $\deg v$ denotes the degree of $v$ in $T$.
Then $\mathbf{1}^T \mathbf{m} = \sum_{v \in V} (2-\deg v) = 2$.
\end{prop}

Let $\mathbf{1}$ denote the all-ones vector.
When we choose a subset $S \subset V(G)$, we no longer have a single ``obvious'' replacement for $\mathbf{m}$ inside $\RR^S$.
Instead, we can take an average over $S$-rooted spanning forests.

In the outline above, our first goal is to find a ``special'' vector $ \mathbf{m}\in \RR^S$ satisfying $D[S] \mathbf{m} = \lambda \mathbf{1}$.
We can approach this first goal as follows: 
consider $\RR^S$ inside the larger vector space $\RR^V = \RR^S \oplus \RR^{V\setminus S}$,
and we wish to find vectors $\mathbf{n}_i \in \RR^{V}$ satisfying
$\pi_S( D \mathbf{n}_i) = \lambda_i \mathbf{1}$.
By finding sufficiently many such vectors $\mathbf{n}_i$,
we can hope to find a linear combination that lies inside $\RR^S \oplus \{0\}$.
% $D \mathbf{n} = (\lambda \mathbf{1}_S) \oplus (-)$.

\begin{prop}
Suppose $v \in V \setminus S$.
For each $s \in S$, let $\mu(v,s) = $ current flowing to $s$
when unit current enters $G$ at $v$ and $G$ is grounded at $S$.
Explicitly,
\begin{equation*}
\mu(v,s) =  \frac{\text{\# of $S$-rooted spanning forests of $G$ whose $s$-component contains $v$}}{\text{\# of $S$-rooted spanning forests of $G$}}
% = (\kappa(..))/(\kappa(G/S))
\end{equation*}
$$
= \frac{\sum_{\trees(G/S)} \mathds{1}(v \in T(s))}{\kappa(G/S)}
$$
Consider the vector $\mathbf{n} \in \RR^V$ defined by
$$
\mathbf{n}(v) = 1,\qquad
\mathbf{n}(s) = -\mu(v,s) \text{ if }s \in S, \qquad
\mathbf{n}(w) = 0 \text{ if } w \not\in S \cup v
$$
Then $\pi_S( D \mathbf{n}) = \lambda \mathbf{1}$ for some $\lambda$.
\end{prop}
\begin{proof}[Proof sketch]
For any $s, s' \in S$, consider tracking value of $D \mathbf{n}$ along path from $s$ to $s'$. 
The value of $D \mathbf{n}$ changes according to current flow in the corresponding system, i.e. $D \mathbf{n}$ records electrical potential.
By assumption $S$ is grounded, so $D\mathbf{n}$ takes the same value at $s$ and $s'$.
\end{proof}


Note that we can express the tree distance $d(v,w)$ as a sum over edges
\begin{equation*}
d(v,w) = \sum_{e \in E(G)} \delta(e; v,w)
\qquad\text{where } \delta(e; v,w) = \begin{cases}
1 &\text{if $e$ lies on $v\sim w$ path}, \\
%\text{if }v,w \text{ are on opposite sides of }e, \\
0 &\text{otherwise}.
\end{cases}
\end{equation*}
\begin{itemize}
\item 
If we fix $v$ and $w$, then $\delta(-;v,w) : E(G) \to \{0,1\}$
is the indicator function for the unique $v~ w$ path in $G$.

\item 
On the other hand if we fix $e$ and $v$,
then the deletion $G\setminus e$ has two connected components,
and $\delta(e;v, -) : V(G) \to \{0,1\}$ 
is the indicator function for the component of $G \setminus e$ not containing $v$.
\end{itemize}


\begin{thm}
Let $G$ be a tree, $S$ a subset of vertices,
and $D[S]$ the corresponding minor of the distance matrix.
Suppose $\mathbf{m}_S \in \RR^{S}$ is defined by
\begin{equation*}
\mathbf{m}_S(v) = \sum_{T \in \mathcal T(G/S)} \sum_{w \in T_v} (2 - \deg w) 
= \sum_{T \in \trees(G/S)} 2 - c(T,v) .
\end{equation*}
Then
$D[S] \mathbf{m} = \lambda \mathbf{1}$
for some constant $\lambda$.
\end{thm}
\begin{proof}
Note that
$$
\mathbf{m}
= \kappa(G/S) \left( \sum_{} + 
\sum_{v \in V \setminus S} ( \deg v - 2) \mathbf{n}_v 
= \right)
$$
\end{proof}

\begin{thm}
With $\mathbf{m}$ defined as above,
$D[S] \mathbf{m} = \lambda \mathbf{1}$
for 
\begin{equation}
\lambda = \sum_{\trees(G/S)} w(T) \sum_{E(G)} \ell(e) - \sum_{\forests_2(G/S)} w(F) k(F,*)^2 .
\end{equation}
\end{thm}
where $c(T,w)$ is the ``cut index'' of the $w$-component of $T$ (as a spanning forest).
\begin{proof}
We have
\begin{align*}
 (D[S] \mathbf{m})(v) &= \sum_{s \in S} d(v,s) \mathbf{m}(s) \\
 &= \sum_{s \in S} \left( \sum_{e \in E(G)} \ell(e) \delta(e; v,s) \right) \left( \sum_{T \in \mathcal T(G/S)} w(T) (2 - \deg^o(T,s)) \right) \\
 &= \sum_{T } \sum_{e} \ell(e) w(T) \sum_{s \in S} (2 - \deg^o(T, s)) \delta(e; v,s) \\
 &= \sum_{T } \sum_{e} \ell(e) w(T) \sum_{s \in S^*(e,v)} (2 - \deg^o(T, s)) .
\end{align*}
where
$$
S^*(e,v) = \{ s \in S : \text{$e$ lies on path from $v$ to $s$}\}.
$$
If $e \in \conv(G,S)$, then $S^*(e,v)$ is nonempty and we have
\begin{align*}
\sum_{s \in S^*(e,v)}(2 - c(T,s)) = \begin{cases}
1 &\text{if }e \not \in T, \\
1 - (2 - c(T \setminus e, *)) &\text{if } e \in T(s'),\, s'\in S^*(e,v) , \\
1 + (2 - c(T \setminus e, *)) &\text{if } e \in T(s'),\, s' \not\in S^*(e,v)
\end{cases}
\end{align*}
Here $c(T\setminus e, *)$ refers to the cut index of the ``free'' component of the spanning forest $T \setminus e$.

If $e \not\in \conv(G,S)$ on the other hand,
$S^*(e,v)$ is empty and we have
$$ \sum_{s \in S^*(e,v)} 2 - c(T,w) = 0 = 1 - (2 - \deg^o(T\setminus e,*))^2,$$
since $\deg^o(T\setminus e,*) = 1$ in this case. 
From \eqref{above} we have
\begin{align*}
(D[S] \mathbf{m})(v) &= \sum_{e} \sum_{T }  \ell(e) w(T) ( 1 - f(v,e,T) )\\
\end{align*}
where
$$
f(v,e,T) = \begin{cases}
0 &\text{if } e\not\in T\\
2 - \deg^o(T\setminus e,*) &\text{if }  e \in T(s')\text{ for some } s'\not\in S^*(e,v) \\
-(2-\deg^o(T\setminus e,*) &\text{if }  e \in T(s')\text{ for some } s'\in S^*(e,v) \\
\end{cases}
$$
(Note that $e \not \in \conv(S,G)$ implies $e \in T$.)

\begin{align*}
(D[S] \mathbf{m})(v) - \sum_{e} \sum_{T }  \ell(e) w(T) 
&= - \sum_T \sum_{e \in T\setminus T(S^*)} \ell(e) w(T) ( 2 - \deg^o(T\setminus e,*)) \\
&\qquad + \sum_T \sum_{e  \in T(S^*)} \ell(e) w(T) (2 - \deg^o(T\setminus e,*) )
\end{align*}

The deletion $T \setminus e$ is an $(S,*)$-rooted spanning forest of $G$,
so we may rewrite the above expression in terms of $\forests_2(G/S)$.
\begin{align*}
(1) &= - \sum_{F \in \forests_2} w(F) (2 - \deg^o(F,*)) \sum_{T} \sum_{e \in T\setminus T(S^*)} \one(F = T \setminus e)  \\
&\qquad +  \sum_{F \in \forests_2} w(F) (2 - \deg^o(F,*)) \sum_{T} \sum_{e \in T(S^*)} \one(F = T \setminus e) 
\end{align*}
Finally, we note that $F = T \setminus e$ is equivalent to $T = F \cup e$, and in particular this only occurs when we choose $e \in \partial F(*)$:
\begin{align*}
(1) &= - \sum_{F \in \forests_2} w(F) (2 - \deg^o(F,*))  \sum_{e \in \partial F(*)} \one( \delta(e; v, F(*)) = 1)  \\
&\qquad +  \sum_{F \in \forests_2} w(F) (2 - \deg^o(F,*)) \sum_{e\in \partial F(*)} \one(\delta(e; v, F(*))=0) 
\end{align*}
Finally, we observe that for any forest $F$,
$$
\#\{e : e \in \partial F(*), \,\delta(e;v, F(*)) = 0 \} = \deg^o(F,*) - 1
$$
and
$$
\#\{e : e \in \partial F(*), \,\delta(e;v, F(*)) = 1 \} = 1
$$
Thus
\begin{align*}
(1) 
&= - \sum_{F \in \forests_2} w(F) (2 - \deg^o(F,*))  (1)  \\
&\qquad +  \sum_{F \in \forests_2} w(F) (2 - \deg^o(F,*)) (\deg^o(F,*) - 1) \\
&= - \!\!\!\sum_{F \in \forests_2(G/S)} w(F) (2 - \deg^o(F,*))^2 .
\end{align*}

The set $\forests_2(G/S)$ of $(S,*)$-rooted spanning forests of $G$ can be partitioned into two types: ``active'' and ``inactive''.
$$
\forests_2(G/S) = \forests_2^{in}(G/S) \sqcup \forests_2^{out}(G/S),
$$
where
$$
\forests_2^{in}(G/S)  = \{ F \in \forests_2(G/S) \text{ such that } \deg^o(*,F) \geq 2\},
$$
$$
\forests_2^{out}(G/S)  = \{ F \in \forests_2(G/S) \text{ such that } \deg^o(*,F) = 1\}.
$$
Moreover, for a given spanning forest
$F \in \forests_2(G/S, *)$,
there are exactly $c(F, *)$ choices of pairs $(T,e) \in \trees(G/S) \times E(G)$ such that
$$
F = T \setminus e.
$$
Consider the map
$$
E(G) \times \trees(G/S) \to \forests_2(G/S)
$$
defined by ...
$$
(e, T) \mapsto T \setminus e .
$$
For a forest $F$ in $\forests_2(G/S)$,
the preimage under this map has $c(F,*)$ elements.


Therefore
(let $\trees = \trees(G/S)$)
\begin{align*}
\sum_{T \in \trees} \sum_{e \in E} \left( \cdots)   \right) 
&= \sum_{T \in \trees} \sum_{e \in E(G)}1 + \sum_{T \in \trees} \sum_{e \in E}\left(  ...  \right)  \\
&= \sum_{T \in \trees} \sum_{e \in E} 1 - \sum_{T \in \trees} \sum_{e \in E} \mathds{1}( e \in T(v) ) (2 - c(T \setminus e, *)) \\
& \qquad\qquad + \sum_{T \in \trees} \sum_{e \in E} \mathds{1}( e \in T \setminus T(v) ) (2 - c(T \setminus e, *)) \\
&= \sum_{T \in \trees} \sum_{e \in E} 1 - \sum_{T \in \trees} \sum_{e \in E} \mathds{1}( e \in T(v) ) \sum_{F \in \forests_2} \mathds{1}(F = T \setminus e)(2 - c(F, *)) \\
& \qquad\qquad + \sum_{T \in \trees} \sum_{e \in E} \mathds{1}( e \in T \setminus T(v) ) \sum_{F \in \forests_2} \mathds{1}(F = T \setminus e)(2 - c(F, *)) \\
&= \sum_{F \in \forests_2} (2- c(F,*)) \sum_{T\in \trees} \sum_{e \in E} \one (F=T\backslash e) (\one(e \in T \setminus T(v)) - \one(e \in T(v)) ) \\
&= \sum_{F \in \forests_2} (2- c(F,*)) \sum_{e \in E} \one (e \not\in F) (\one(e \not\in (F \cup e)(v)) - \one(e \in (F\cup e)(v)) ) \\
&= \sum_{F \in \forests_2} (2- c(F,*)) ((c(F,*)-1) - 1) \\
&= - \sum_{F \in \forests_2} (2- c(F,*))^2 .
\end{align*}
\end{proof}

\begin{prop}
Let $G = (V,E)$ be a tree, and $S \subset V$.
Suppose we label $S = \{s_1, \ldots, s_r\}$
and $V \setminus S = \{t_1, \ldots, t_{n-r}\}$.
For each $t_i \in V\setminus S$,
consider
$\mathbf{f}_i  \in \RR^V$
defined by
\end{prop}

\begin{eg}
If 
$$
D[S \cup t] = \begin{bmatrix}
0 & a & b & c \\
a & 0 & a + b & a + c \\
b & a + b & 0 & b + c \\
c & a + c & b + c & 0
\end{bmatrix}
$$
then
$$
 \begin{bmatrix}
0 & a & b & c \\
a & 0 & a + b & a + c \\
b & a + b & 0 & b + c \\
c & a + c & b + c & 0
\end{bmatrix}
\begin{bmatrix}
ab + ac + bc \\ -bc \\ -ac \\ -ab 
\end{bmatrix}
= \begin{bmatrix}
-3abc \\ -abc \\ -abc \\ -abc
\end{bmatrix}
$$
\end{eg}

\section{Physical interpretation}

If we consider $G$ as a network of wires with each edge containing a unit resistor,
which is grounded at all nodes in $S$,
then $\mathbf{m}_S$ records the currents flowing to $S$
when current is added on $V\setminus S$ in the amount $2 - \deg v$
for each $v\not\in S$.

\section{Examples}

\begin{eg}
Suppose $G$ is a tree consisting of three paths joined at a central vertex.
Let $S$ consist of the central vertex, and the three endpoints of the paths. 
The corresponding minor of the distance matrix is
$$
D[S] = \begin{bmatrix}
0 & a & b & c \\
a & 0 & a + b & a + c \\
b & a + b & 0 & b + c \\
c & a + c & b + c & 0
\end{bmatrix}
\sim \begin{bmatrix}
0 & a & b & c \\
a & -a & a  & a \\
b & b & -b & b \\
c & c & c & -c
\end{bmatrix}
\sim \begin{bmatrix}
0 & a & b & c \\
a & -2a & 0 & 0 \\
b & 0 & -2b & 0 \\
c & 0 & 0 & -2c
\end{bmatrix}.
$$
The determinant is
\begin{align*}
\det D[S] 
= -4(a+b+c)abc.
\end{align*}
\end{eg}

\begin{eg}
Suppose $\Gamma$ is a tripod with lengths $a,b,c$ and corresponding leaf vertices $u,v,w$.
\[
\begin{tikzpicture}
	\draw (0,0) -- (-2,0);
	\draw (2,1) -- (0,0) -- (2,-1);
	\filldraw[black] (-2,0) circle (2pt);
	\filldraw[black] (2,1) circle (2pt);
	\filldraw[black] (2,-1) circle (2pt);

	\node[above] at (-1,0) {$a$};
	\node[above] at (1,0.5) {$b$};
	\node[below] at (1,-0.5) {$c$};

	\node[left] at (-2,0) {$u$};
	\node[right] at (2,1) {$v$};
	\node[right] at (2,-1) {$w$};
\end{tikzpicture}
\]
Let $B = \{u,v,w\}$.
Then 
$$
D[B] = \begin{bmatrix}
0 & a + b & a + c \\
a + b & 0 & b + c \\
a + c & b + c & 0
\end{bmatrix}.
$$
and
\begin{align*}
\det D[B] &= 2(a+b)(a+c)(b+c) 
= 2\left( (a+b+c)(ab + ac + bc) - abc \right).
\end{align*}
The ``special vector'' in this example is $\mathbf{m}^T = \begin{bmatrix} a(b + c) & b(a + c) & c(a + b) \end{bmatrix}$.
\end{eg}

\section{Further work}

See \cite{richman-shokrieh-wu}.

\section*{Acknowledgements}
The authors would like to thank Ravindra Bapat for helpful discussion.


\bibliography{tree-distance-ref} 
\bibliographystyle{abbrv}

\end{document}
