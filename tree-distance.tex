%v1
\documentclass{amsart}
\usepackage{amssymb, amsmath, amsfonts, amsthm}
\usepackage{graphics, enumerate, mathrsfs, mathtools,tikz-cd,soul,csquotes}
\usetikzlibrary{positioning,arrows,scopes}
%\usepackage[left=1.25in,
%			right=1.25in,
%			top=1.25in,
%			bottom=1.25in]{geometry}
%\usepackage{fancyhdr}
%\pagestyle{fancy}

% COMMENT OUT FOR FINAL VERSION
\usepackage{showkeys}

\newtheorem{thm}{Theorem}
\newtheorem*{thm*}{Theorem}
\newtheorem{obs}[thm]{Observation}
\newtheorem{prop}[thm]{Proposition}
\newtheorem{lem}[thm]{Lemma}
\newtheorem{cor}[thm]{Corollary}

\theoremstyle{definition}
\newtheorem{prob}[thm]{Problem}
\newtheorem{dfn}[thm]{Definition}
\newtheorem{eg}[thm]{Example}
\newtheorem{rmk}[thm]{Remark}
\newtheorem{conj}[thm]{Conjecture}

\newcommand{\CC}{\mathbb{C}}
\newcommand{\FF}{\mathbb{F}}
\newcommand{\RR}{\mathbb{R}}
\newcommand{\ZZ}{\mathbb{Z}}
\newcommand{\QQ}{\mathbb{Q}}
\newcommand{\NN}{\mathbb{N}}
\newcommand{\PP}{\mathbb{P}}
\newcommand{\cO}{\mathcal{O}}
\newcommand{\cU}{\mathcal{U}}
\newcommand{\cL}{\mathcal{L}}
\newcommand{\cC}{\mathcal{C}}
\newcommand{\cE}{\mathcal{E}}

%\newcommand{\energy}{\cE}
\DeclareMathOperator{\energy}{\cE}
\DeclareMathOperator{\supp}{supp}
%%%% capacity
\DeclareMathOperator{\Capacity}{\textsc{cap}}
\newcommand{\posCap}{c_{pos}}
\DeclareMathOperator{\argmax}{argmax}
\DeclareMathOperator{\argmin}{argmin}
\DeclareMathOperator{\val}{val}
%topological interior
\DeclareMathOperator{\interior}{int}

\DeclareMathOperator{\trop}{trop}
\DeclareMathOperator{\rspan}{row.span}
\DeclareMathOperator{\Sym}{Sym}
\DeclareMathOperator{\Div}{Div}
\DeclareMathOperator{\Eff}{Eff}
\DeclareMathOperator{\indeg}{indeg}
\DeclareMathOperator{\Pic}{Pic}
\DeclareMathOperator{\Jac}{Jac}
%\DeclareMathOperator{\ker}{ker}
\DeclareMathOperator{\coker}{coker}
\DeclareMathOperator{\PL}{PL}
%%% use either \Delta or Div %%%
\DeclareMathOperator{\Divisor}{\Delta}
\DeclareMathOperator{\zspan}{span}
\DeclareMathOperator{\im}{im}
\DeclareMathOperator{\Br}{Brk}
\DeclareMathOperator{\ev}{ev}
\DeclareMathOperator{\vol}{vol}
\DeclareMathOperator{\coeff}{coeff}

\renewcommand{\Re}{\mathrm{Re}}
\newcommand{\dsp}{\displaystyle}
\newcommand{\pderiv}[1]{\frac{\partial}{\partial{#1}}}
\newcommand{\angles}[1]{\langle {#1} \rangle}

% for comments
\newcommand{\note}[1]{{\color{red} \sf $\diamondsuit$  {#1} $\diamondsuit$ }}
\newcommand{\todo}[1]{\note{#1}}
\newcommand{\harry}[1]{{\color{red} \sf $\diamondsuit$  Harry: {#1} $\diamondsuit$ }}

\begin{document}
\title[Tree distance minors]{Minors of tree distance matrices}
\author{Harry Richman}
\author{Farbod Shokrieh}
\author{Chenxi Wu}
\date{v1, \today  \,(Preliminary draft, not for circulation).}
%\thanks{This work was partially supported by ....}


\begin{abstract}
We prove a formula for the determinant of 
a principal minor of
the distance matrix of a weighted tree.pr
\end{abstract}
\maketitle

\setcounter{tocdepth}{1}
\tableofcontents

\section{Introduction}

\cite{graham-hoffman-hosoya}

Special case: $\Gamma$ is a metric tree.
\begin{thm}\label{thm:max-measure}
Suppose $\Gamma = (G,\ell)$ is a metric tree,
and $A \subset V(G)$ a finite subset of vertices.
The measure in $Meas^1(A)$
which maximizes distance energy is given by
\begin{equation}\label{eq:max-measure}
\mu_A = \frac1{\displaystyle \sum_{\mathcal F} w(F)} \sum_{\mathcal F} w(F) \sum_{x \in A}  \mu_{can}(F_x) \,\delta_x
= \sum_{x\in A} \left( \frac{\sum_{\mathcal F} w(F) \mu_{can}(F_x)}{\sum_{\mathcal F} w(F)} \right) \delta_x
\end{equation}
where $\delta_x$ is the Dirac measure at $x$,
$\mathcal F = \mathcal F(A,G)$ denotes the set of $A$-rooted spanning forests of $G$,
$w(F)= \prod_{e \not \in F} \ell(e) $
is the weight of $F$,
$F_x$ denotes the $x$-component of the forest $F$
and
$$
\mu_{can}(F_x) = \sum_{y \in V(F_x)} \frac{2 - \val(y)}{2}.
$$
\end{thm}

Equivalently, the optimal measure is
$$
\mu_A = \sum_{x \in A} \delta_x\left( \frac{2 - \val(x)}{2} +  \sum_{y \in \Gamma \setminus A} \frac{2 - \val(y)}{2} c(y,x;\Gamma) \right)
$$
where
% $\delta_x$ denotes the Dirac measure at $x$,
$c(y,x;\Gamma)$ denotes the current to $x \in A$ when unit current flows from $y$ to $A$.
In particular,
\begin{equation}
c(y,x;\Gamma) = \frac{\sum_F \epsilon(y,x,F) w(F)}{ \sum_F w(F)}
\end{equation}
where both sums are taken over $A$-rooted spanning forests $F$ of $\Gamma$,
%$w(F) = \prod_{e \not \in F} \ell(e) $
%is the weight of $F$,
and
\begin{equation*}
\epsilon(y,x,F) = \begin{cases}
1 &\text{if $x$-component of $F$ contains $y$} \\
0 &\text{otherwise.}
\end{cases}
\end{equation*}

\begin{thm}\label{thm:max-capacity}
Suppose $\Gamma = (G,\ell)$ is a metric tree, $A \subset V(G)$ is a finite subset,
and $\mu_A$ is the measure \eqref{eq:max-measure}.
Then
\begin{equation}\label{eq:max-capacity}
\energy(\mu_A) = \frac14 \ell(\Gamma) - \left( \frac{ \sum_{\mathcal F^*} \mu_{can}(F_{2,*})^2 w(F_2)}{\sum_{\mathcal F} w(F_1) } \right).
\end{equation}
where 
$F_1$ varies over $A$-rooted spanning forests of $G$,
$F_2$ varies over all $(A,*)$-rooted spanning forests of $G$,
$F_{2,*}$ denotes the $*$-component of $F_2$, 
and
\begin{equation}
\mu_{can}(F_{2,*}) 
= \frac12 \sum_{x \in V( F_{2,*})} {2 - \val_\Gamma(x)}.
\end{equation}
\end{thm}
In other words, 
$\mu_{can}$ denotes the canonical measure on $\Gamma$.


\subsection{Previous work} 



\subsection{Notation}

$\Gamma$ a compact metric graph

$G$ a finite graph, 
loops and parallel edges allowed,
possibly disconnected

$E(G)$ edge set of $G$

$V(G)$ vertex set of $G$

$(G,\ell)$ a combinatorial model for a metric graph,
where 
%$G$ is a combinatorial graph and 

$\ell : E(G) \to \RR_{>0}$
is a length function on edges of $G$

$\cC(\Gamma)$ continuous $\RR$-valued functions on $\Gamma$

$Meas(\Gamma)$ signed Borel measures on $\Gamma$

$Meas_{\geq 0}(\Gamma)$ positive Borel measures on $\Gamma$

\section{Background}


\section{Proofs}



\section{Examples}
\begin{eg}
Suppose $\Gamma$ is a tripod with lengths $a,b,c$ and corresponding leaf vertices $u,v,w$.
\[
\begin{tikzpicture}
	\draw (0,0) -- (-2,0);
	\draw (2,1) -- (0,0) -- (2,-1);
	\filldraw[black] (-2,0) circle (2pt);
	\filldraw[black] (2,1) circle (2pt);
	\filldraw[black] (2,-1) circle (2pt);

	\node[above] at (-1,0) {$a$};
	\node[above] at (1,0.5) {$b$};
	\node[below] at (1,-0.5) {$c$};

	\node[left] at (-2,0) {$u$};
	\node[right] at (2,1) {$v$};
	\node[right] at (2,-1) {$w$};
\end{tikzpicture}
\]
Let $A = \{u,v,w\}$.
Then 
\begin{align*}
\mu_A &= \frac{ab(\frac12\delta_u + \frac12\delta_v) + ac(\frac12\delta_u + \frac12\delta_w) + bc(\frac12\delta_v + \frac12\delta_w) }{ab + ac + bc} \\
&= \frac12 \left( \frac{ab + ac}{ab + ac + bc}\delta_u + \frac{ab + bc}{ab + ac + bc} \delta_v + \frac{ac + bc}{ab + ac + bc} \delta_w \right) .
\end{align*}
\end{eg}


\section*{Acknowledgements}
The authors would like to thank ...


\bibliography{tree-distance-ref} 
\bibliographystyle{abbrv}

\end{document}
