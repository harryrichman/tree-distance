%v1
\documentclass{amsart}
\usepackage{amssymb, amsmath, amsfonts, amsthm}
\usepackage{graphics, enumerate, mathrsfs, mathtools,tikz-cd,soul,csquotes}
\usetikzlibrary{positioning,arrows,scopes}
%\usepackage[left=1.25in,
%			right=1.25in,
%			top=1.25in,
%			bottom=1.25in]{geometry}
%\usepackage{fancyhdr}
%\pagestyle{fancy}

% COMMENT OUT FOR FINAL VERSION
\usepackage{showkeys}

\newtheorem{thm}{Theorem}
\newtheorem*{thm*}{Theorem}
\newtheorem{obs}[thm]{Observation}
\newtheorem{prop}[thm]{Proposition}
\newtheorem{lem}[thm]{Lemma}
\newtheorem{cor}[thm]{Corollary}

\theoremstyle{definition}
\newtheorem{prob}[thm]{Problem}
\newtheorem{dfn}[thm]{Definition}
\newtheorem{eg}[thm]{Example}
\newtheorem{rmk}[thm]{Remark}
\newtheorem{conj}[thm]{Conjecture}

\newcommand{\CC}{\mathbb{C}}
\newcommand{\FF}{\mathbb{F}}
\newcommand{\RR}{\mathbb{R}}
\newcommand{\ZZ}{\mathbb{Z}}
\newcommand{\QQ}{\mathbb{Q}}
\newcommand{\NN}{\mathbb{N}}
\newcommand{\PP}{\mathbb{P}}
\newcommand{\cO}{\mathcal{O}}
\newcommand{\cU}{\mathcal{U}}
\newcommand{\cL}{\mathcal{L}}
\newcommand{\cC}{\mathcal{C}}
\newcommand{\cE}{\mathcal{E}}

% sum of cofactors
\DeclareMathOperator{\cof}{cof}
\DeclareMathOperator{\energy}{\cE}
\DeclareMathOperator{\supp}{supp}
%%%% capacity
\DeclareMathOperator{\Capacity}{\textsc{cap}}
\newcommand{\posCap}{c_{pos}}
\DeclareMathOperator{\argmax}{argmax}
\DeclareMathOperator{\argmin}{argmin}
\DeclareMathOperator{\val}{val}
%topological interior
\DeclareMathOperator{\interior}{int}

\DeclareMathOperator{\trop}{trop}
\DeclareMathOperator{\rspan}{row.span}
\DeclareMathOperator{\Sym}{Sym}
\DeclareMathOperator{\Div}{Div}
\DeclareMathOperator{\Eff}{Eff}
\DeclareMathOperator{\indeg}{indeg}
\DeclareMathOperator{\Pic}{Pic}
\DeclareMathOperator{\Jac}{Jac}
%\DeclareMathOperator{\ker}{ker}
\DeclareMathOperator{\coker}{coker}
\DeclareMathOperator{\PL}{PL}
%%% use either \Delta or Div %%%
\DeclareMathOperator{\Divisor}{\Delta}
\DeclareMathOperator{\zspan}{span}
\DeclareMathOperator{\im}{im}
\DeclareMathOperator{\Br}{Brk}
\DeclareMathOperator{\ev}{ev}
\DeclareMathOperator{\vol}{vol}
\DeclareMathOperator{\coeff}{coeff}

\renewcommand{\Re}{\mathrm{Re}}
\newcommand{\dsp}{\displaystyle}
\newcommand{\pderiv}[1]{\frac{\partial}{\partial{#1}}}
\newcommand{\angles}[1]{\langle {#1} \rangle}

% for comments
\newcommand{\note}[1]{{\color{red} \sf $\diamondsuit$  {#1} $\diamondsuit$ }}
\newcommand{\todo}[1]{\note{#1}}
\newcommand{\harry}[1]{{\color{red} \sf $\diamondsuit$  Harry: {#1} $\diamondsuit$ }}

\begin{document}
\title[Tree distance minors]{Minors of tree distance matrices}
\author{Harry Richman}
\author{Farbod Shokrieh}
\author{Chenxi Wu}
\date{v1, \today  \,(Preliminary draft, not for circulation).}
%\thanks{This work was partially supported by ....}


\begin{abstract}
We prove a formula for the determinant of 
a principal minor of
the distance matrix of a weighted tree.
This generalizes a result of Graham and Pollak.
\end{abstract}
\maketitle

\setcounter{tocdepth}{1}
\tableofcontents

\section{Introduction}

Suppose $T$ is a tree with $n$ vertices and $m=n-1$ edges.
Let $D$ denote the distance matrix of $T$.
In
\cite{graham-pollak},
Graham and Pollak proved that
\begin{equation}
\det(D) = (-1)^{n-1} 2^{n-2} (n-1). 
\end{equation}

A weighted version was proved by Bapat--Kirkland--Neumann~\cite{bapat-kirkland-neumann}.
\begin{thm}
\label{thm:main}
Suppose $G$ is a tree with $n$ vertices, and $S \subset V(G)$ is a subset of vertices.
Let $D$ denote the distance matrix of $G$,
and $D[S]$ the principal minor that includes the $S$-indexed rows and columns.
Then
\begin{equation}\label{eq:main}
\det D[S] = (-1)^{|S|-1} 2^{|S|-2} \left( (n-1)\, \kappa(G/S)  - \sum_{\mathcal F_2(G/S)} k(F_{*})^2  \right).
\end{equation}
where 
$G/S$ denotes the quotient graph that identifies together vertices in $S$,
$\mathcal F_2$ is the set of two-component spanning forests,
$F_{*}$ denotes the $*$-component of $F$, 
and
\begin{equation*}
k(F_{*}) 
= \sum_{x \in V( F_{*})} {2 - \deg(x)} = 2 - c(F,*).
\end{equation*}
\end{thm}

Weighted version:
\begin{thm}
\label{thm:w-max-capacity}
Suppose $G$ is a finite, weighted tree, and $A \subset V(G)$ is a subset of vertices.
Then
\begin{equation}\label{eq:w-max-capacity}
\det D[A] = (-1)^{|S|-1} 2^{|S|-2} \left( \sum_{e}\ell(e) \sum_{\mathcal T} w(T) - \sum_{\mathcal F^*} k(F_{2,*})^2 w(F_2) \right).
\end{equation}
where 
$\mathcal T(G/A)$ denotes the set of $A$-rooted spanning forests of $G$,
$F_2$ varies over all $(A,*)$-rooted spanning forests of $G$,
$F_{2,*}$ denotes the $*$-component of $F_2$.
\end{thm}


\begin{thm}[Monotonicity of principal minors]
Suppose $G = (V,E)$ is a finite, weighted tree with distance matrix $D$.
If $A,B \subset V(G)$
are vertex subsets with
$A \subset B$,
then
\begin{equation*}
\left| \frac{\det D[A]}{\cof D[A]} \right| \leq \left| \frac{\det D[B]}{\cof D[B]} \right| .
\end{equation*}
\end{thm}

\begin{thm}[Nonsingular minors]
Let $G$ be a finite, weighted tree
with distance matrix $D$,
and let $S \subset V(G)$ be a subset of vertices.
If $|S|\geq 2$ then $\det D[S] \neq 0$.
\end{thm}

\subsection{Previous work} 

The following theorem is due to Kirchhoff.
For any graph $G$, let $\kappa(G)$ denote the number of spanning trees of $G$.
\begin{thm}[All-minors matrix tree theorem]
Let $G = (V,E)$ be a finite graph.
Let $L$ denote the Laplacian matrix of $G$.
Then for any nonempty vertex set $S \subset V(G)$,
\begin{equation}
\det L[V \setminus S] = \kappa( G / S) .
\end{equation}
\end{thm}
Note that $\kappa(G/S)$ is also the number of $S$-rooted spanning forests of $G$.


\begin{thm}[\cite{bapat-sivasubramanian}]
Let $T$ be a tree with $m+1$ vertices and $m$ edges.
Let $D$ be the distance matrix of $T$, and $L$ the Laplacian matrix.
Let $S \subset V(T)$ be a subset of vertices of $T$. 
Then
\begin{equation*}
\cof D[S] = (-2)^{|S|-1} \det L[V \setminus S] .
\end{equation*}
\end{thm}

\subsection{Notation}

% $\Gamma$ a compact metric graph

$G$ a finite graph, 
loops and parallel edges allowed,
possibly disconnected

$E(G)$ edge set of $G$

$V(G)$ vertex set of $G$

%$\ell : E(G) \to \RR_{>0}$
%is a length function on edges of $G$

% $\cC(\Gamma)$ continuous $\RR$-valued functions on $\Gamma$

% $Meas(\Gamma)$ signed Borel measures on $\Gamma$

% $Meas_{\geq 0}(\Gamma)$ positive Borel measures on $\Gamma$

\section{Background}


\section{Proofs}

Outline of proof: given subset $S$ and distance matrix minor $D[S]$, we will
\begin{enumerate}
\item 
Find vector $\mathbf{m}$ such that $D[S]\mathbf{m} = \lambda \mathbf{1}$.

\item 
Compute the sum of entries of $\mathbf{m}$, i.e. $\mathbf{1}^T\mathbf{m}$.

\item 
Note the identity
$$ \mathbf{1}^T \mathbf{m} 
= \lambda (\mathbf{1}^T D[S]^{-1} \mathbf{1}) 
= \lambda \frac{\cof D[S]}{\det D[S]} .$$
where $\cof D[S]$ is the sum of cofactors of $D[S]$.

\item 
Use known expression for $\cof D[S]$ to compute
$$
\det D[S] = \lambda (\cof D[S]) \left( \mathbf{1}^T \mathbf{m} \right)^{-1}.
$$
\end{enumerate}
The interesting part will be hidden in the constant $\lambda$.

\begin{eg}
Suppose $G$ is a tree consisting of three paths joined at a central vertex.
Let $S$ consist of the central vertex, and the three endpoints of the paths. 
The corresponding minor of the distance matrix is
$$
D[S] = \begin{bmatrix}
0 & a & b & c \\
a & 0 & a + b & a + c \\
b & a + b & 0 & b + c \\
c & a + c & b + c & 0
\end{bmatrix}.
$$
Following the steps outlined above:
\begin{enumerate}
\item 
The vector $\mathbf{m} = \begin{bmatrix} -1 \\ 1 \\ 1 \\ 1 \end{bmatrix}$ satisfies
$
D[S] \mathbf{m} = (a+b+c) \mathbf{1}
$

\item 
The sum of entries of $\mathbf{m}$ is $\mathbf{1}^T \mathbf{m} = 2$.

\item 
We have 
$$2 = \mathbf{1}^T \mathbf{m} = \lambda (\mathbf{1}^T D[S] \mathbf{1}) = \displaystyle \lambda \frac{\cof D[S]}{\det D[S]}.$$

\item 
The cofactor sum is 
$\cof D[S] = -8 abc$,
so the determinant is
\begin{align*}
\det D[S] 
= \lambda \frac{\cof A}{\mathbf{1}^T \mathbf{m}}
= (a+b+c) (-8 abc)\frac{1}{2}
= -4(a+b+c)abc.
\end{align*}

\end{enumerate}
\end{eg}

\begin{prop}
Let $T = (V,E)$ a tree, and consider the vector $\mathbf{m} \in \RR^V$
defined by 
\begin{equation*}
\mathbf{m}(v) = 2 - \deg v,
\end{equation*}
where $\deg v$ denote the degree of $v$ in $T$.
Then $\mathbf{1}^T \mathbf{m} = \sum_{v \in V} (2-\deg v) = 2$.
\end{prop}

Let $\mathbf{1}$ denote the all-ones vector.
\begin{thm}
Let $T$ be a tree, $S$ a subset of vertices,
and $D[S]$ the corresponding minor of the distance matrix.
Suppose $\mathbf{m} \in \RR^{S}$ is defined by
\begin{equation*}
\mathbf{m}(v) = \sum_{T \in \mathcal T(T/S)} \sum_{w \in T_v} (2 - \deg w) 
\end{equation*}
Then
$D[S] \mathbf{m} = \lambda \mathbf{1}$
for some constant $\lambda$.
\end{thm}
\begin{proof}
For any $u,v \in S$, 
we must show that $(D[S] \mathbf{m})(u) = (D[S] \mathbf{m})(v)$.
We have
\begin{align*}
 (D[S] \mathbf{m})(v) &= \sum_{w \in S} d(v,w) \mathbf{m}(w) \\
 &= \sum_{w \in S} d(v,w) \sum_{T \in \mathcal T(G/S)} \sum_{z \in T(w)} (2 - \deg z) \\
 &= \sum_{T \in \mathcal T(G/S)} \sum_{w \in S} \sum_{z \in T(w)} (2 - \deg z) d(v,w) \\
 &= \sum_{T \in \mathcal T(G/S)} \sum_{w \in S} (2 - c(T,w)) d(v,w).
\end{align*}
where $c(T,w)$ is the ``cut index'' of the $w$-component of $T$ (as a spanning forest).

Note that we can express the tree distance $d(v,w)$ as a sum over edges
\begin{equation*}
d(v,w) = \sum_{e \in E(G)} \delta(e; v,w)
\qquad\text{where } \delta(e; v,w) = \begin{cases}
1 &\text{if $e$ lies on $v\sim w$ path}, \\
%\text{if }v,w \text{ are on opposite sides of }e, \\
0 &\text{otherwise}.
\end{cases}
\end{equation*}
Thus
\begin{align*}
(D[S] \mathbf{m})(v) &= \sum_{T \in \mathcal T(G/S)} \sum_{w \in S} (2 - c(T,w))  \sum_{e \in E(G)} \delta(e;v,w) \\
&= \sum_{T \in \mathcal T(G/S)} \sum_{e \in E(G)} \left( \sum_{w \in S}(2 - c(T,w))   \delta(e;v,w) \right)
\end{align*}
\end{proof}


\section{Examples}

\begin{eg}
Suppose $G$ is a tree consisting of three paths joined at a central vertex.
Let $S$ consist of the central vertex, and the three endpoints of the paths. 
The corresponding minor of the distance matrix is
$$
D[S] = \begin{bmatrix}
0 & a & b & c \\
a & 0 & a + b & a + c \\
b & a + b & 0 & b + c \\
c & a + c & b + c & 0
\end{bmatrix}
\sim \begin{bmatrix}
0 & a & b & c \\
a & -a & a  & a \\
b & b & -b & b \\
c & c & c & -c
\end{bmatrix}.
\sim \begin{bmatrix}
0 & a & b & c \\
a & -2a & 0 & 0 \\
b & 0 & -2b & 0 \\
c & 0 & 0 & -2c
\end{bmatrix}.
$$
The determinant is
\begin{align*}
\det D[S] 
= -4(a+b+c)abc.
\end{align*}
\end{eg}

\begin{eg}
Suppose $\Gamma$ is a tripod with lengths $a,b,c$ and corresponding leaf vertices $u,v,w$.
\[
\begin{tikzpicture}
	\draw (0,0) -- (-2,0);
	\draw (2,1) -- (0,0) -- (2,-1);
	\filldraw[black] (-2,0) circle (2pt);
	\filldraw[black] (2,1) circle (2pt);
	\filldraw[black] (2,-1) circle (2pt);

	\node[above] at (-1,0) {$a$};
	\node[above] at (1,0.5) {$b$};
	\node[below] at (1,-0.5) {$c$};

	\node[left] at (-2,0) {$u$};
	\node[right] at (2,1) {$v$};
	\node[right] at (2,-1) {$w$};
\end{tikzpicture}
\]
Let $A = \{u,v,w\}$.
Then 
$$
D[A] = \begin{bmatrix}
0 & a + b & a + c \\
a + b & 0 & b + c \\
a + c & b + c & 0
\end{bmatrix}.
$$
and
\begin{align*}
\det D[A] &= 2(a+b)(a+c)(b+c) 
= 2\left( (a+b+c)(ab + ac + bc) - abc \right).
\end{align*}
The ``special vector'' in this example is $\mathbf{m}^T = \begin{bmatrix} a(b + c) & b(a + c) & c(a + b) \end{bmatrix}$.
\end{eg}


\section*{Acknowledgements}
The authors would like to thank Ravindra Bapat for helpful discussion.


\bibliography{tree-distance-ref} 
\bibliographystyle{abbrv}

\end{document}
