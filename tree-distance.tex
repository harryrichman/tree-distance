%v1
\documentclass{amsart}
\usepackage{amssymb, amsmath, amsfonts, amsthm}
\usepackage{graphics, enumerate, mathrsfs, mathtools,tikz-cd,soul,csquotes}
\usetikzlibrary{positioning,arrows,scopes}
%\usepackage[left=1.25in,
%			right=1.25in,
%			top=1.25in,
%			bottom=1.25in]{geometry}
%\usepackage{fancyhdr}
%\pagestyle{fancy}

% COMMENT OUT FOR FINAL VERSION
\usepackage{showkeys}

\newtheorem{thm}{Theorem}
\newtheorem*{thm*}{Theorem}
\newtheorem{obs}[thm]{Observation}
\newtheorem{prop}[thm]{Proposition}
\newtheorem{lem}[thm]{Lemma}
\newtheorem{cor}[thm]{Corollary}

\theoremstyle{definition}
\newtheorem{prob}[thm]{Problem}
\newtheorem{dfn}[thm]{Definition}
\newtheorem{eg}[thm]{Example}
\newtheorem{rmk}[thm]{Remark}
\newtheorem{conj}[thm]{Conjecture}

\newcommand{\CC}{\mathbb{C}}
\newcommand{\FF}{\mathbb{F}}
\newcommand{\RR}{\mathbb{R}}
\newcommand{\ZZ}{\mathbb{Z}}
\newcommand{\QQ}{\mathbb{Q}}
\newcommand{\NN}{\mathbb{N}}
\newcommand{\PP}{\mathbb{P}}
\newcommand{\cO}{\mathcal{O}}
\newcommand{\cU}{\mathcal{U}}
\newcommand{\cL}{\mathcal{L}}
\newcommand{\cC}{\mathcal{C}}
\newcommand{\cE}{\mathcal{E}}

% sum of cofactors
\DeclareMathOperator{\cof}{cof}
\DeclareMathOperator{\energy}{\cE}
\DeclareMathOperator{\supp}{supp}
%%%% capacity
\DeclareMathOperator{\Capacity}{\textsc{cap}}
\newcommand{\posCap}{c_{pos}}
\DeclareMathOperator{\argmax}{argmax}
\DeclareMathOperator{\argmin}{argmin}
\DeclareMathOperator{\val}{val}
%topological interior
\DeclareMathOperator{\interior}{int}

\DeclareMathOperator{\trop}{trop}
\DeclareMathOperator{\rspan}{row.span}
\DeclareMathOperator{\Sym}{Sym}
\DeclareMathOperator{\Div}{Div}
\DeclareMathOperator{\Eff}{Eff}
\DeclareMathOperator{\indeg}{indeg}
\DeclareMathOperator{\Pic}{Pic}
\DeclareMathOperator{\Jac}{Jac}
%\DeclareMathOperator{\ker}{ker}
\DeclareMathOperator{\coker}{coker}
\DeclareMathOperator{\PL}{PL}
%%% use either \Delta or Div %%%
\DeclareMathOperator{\Divisor}{\Delta}
\DeclareMathOperator{\zspan}{span}
\DeclareMathOperator{\im}{im}
\DeclareMathOperator{\Br}{Brk}
\DeclareMathOperator{\ev}{ev}
\DeclareMathOperator{\vol}{vol}
\DeclareMathOperator{\coeff}{coeff}

\renewcommand{\Re}{\mathrm{Re}}
\newcommand{\dsp}{\displaystyle}
\newcommand{\pderiv}[1]{\frac{\partial}{\partial{#1}}}
\newcommand{\angles}[1]{\langle {#1} \rangle}

% for comments
\newcommand{\note}[1]{{\color{red} \sf $\diamondsuit$  {#1} $\diamondsuit$ }}
\newcommand{\todo}[1]{\note{#1}}
\newcommand{\harry}[1]{{\color{red} \sf $\diamondsuit$  Harry: {#1} $\diamondsuit$ }}

\begin{document}
\title[Tree distance minors]{Minors of tree distance matrices}
\author{Harry Richman}
\author{Farbod Shokrieh}
\author{Chenxi Wu}
\date{v1, \today  \,(Preliminary draft, not for circulation).}
%\thanks{This work was partially supported by ....}


\begin{abstract}
We prove a formula for the determinant of 
a principal minor of
the distance matrix of a weighted tree.
\end{abstract}
\maketitle

\setcounter{tocdepth}{1}
\tableofcontents

\section{Introduction}

Suppose $T$ is a tree with $n$ vertices and $m=n-1$ edges.
let $D$ denote the distance matrix of $T$.
In
\cite{graham-pollak},
Graham and Pollak proved that
\begin{equation}
\det(D) = (-1)^{n-1} 2^{n-2} (n-1). 
\end{equation}

\cite{graham-hoffman-hosoya}, 

Special case: $\Gamma$ is a metric tree.

\begin{thm}\label{thm:max-capacity}
Suppose $G$ is a finite (weighted?) tree, and $A \subset V(G)$ is a subset of vertices.
Then
\begin{equation}\label{eq:max-capacity}
\det D[A] = \frac14 \ell(\Gamma)\left( \sum_{\mathcal T} w(T) \right) - \frac14 \left( \sum_{\mathcal F^*} \mu_{can}(F_{2,*})^2 w(F_2) \right).
\end{equation}
where 
$\mathcal T(G/A)$ denotes the set of $A$-rooted spanning forests of $G$,
$F_2$ varies over all $(A,*)$-rooted spanning forests of $G$,
$F_{2,*}$ denotes the $*$-component of $F_2$, 
and
\begin{equation}
\mu_{can}(F_{2,*}) 
= \sum_{x \in V( F_{2,*})} {2 - \deg(x)}.
\end{equation}
\end{thm}
In other words, 
$\mu_{can}$ denotes the canonical measure on $\Gamma$.


\begin{thm}[Monotonicity of principal minors]
Suppose $G = (V,E)$ is a finite, weighted tree with distance matrix $D$.
If $A,B \subset V$
are vertex subsets with
$A \subset B$,
then
\begin{equation*}
\left| \frac{\det D[A]}{\cof D[A]} \right| \leq \left| \frac{\det D[B]}{\cof D[B]} \right| .
\end{equation*}
\end{thm}

\subsection{Previous work} 

\begin{thm}[\cite{bapat-sivasubramanian}]
Let $T$ be a tree with $m+1$ vertices and $m$ edges.
Let $D$ be the distance matrix of $T$, and $L$ the Laplacian matrix.
Let $S \subset V(T)$ be a subset of vertices of $T$. 
Then
\begin{equation*}
\cof D[S] = (-2)^{|S|-1} \det L[V \setminus S] .
\end{equation*}
\end{thm}

\subsection{Notation}

$\Gamma$ a compact metric graph

$G$ a finite graph, 
loops and parallel edges allowed,
possibly disconnected

$E(G)$ edge set of $G$

$V(G)$ vertex set of $G$

$(G,\ell)$ a combinatorial model for a metric graph,
where 
%$G$ is a combinatorial graph and 

$\ell : E(G) \to \RR_{>0}$
is a length function on edges of $G$

$\cC(\Gamma)$ continuous $\RR$-valued functions on $\Gamma$

$Meas(\Gamma)$ signed Borel measures on $\Gamma$

$Meas_{\geq 0}(\Gamma)$ positive Borel measures on $\Gamma$

\section{Background}


\section{Proofs}



\section{Examples}
\begin{eg}
Suppose $\Gamma$ is a tripod with lengths $a,b,c$ and corresponding leaf vertices $u,v,w$.
\[
\begin{tikzpicture}
	\draw (0,0) -- (-2,0);
	\draw (2,1) -- (0,0) -- (2,-1);
	\filldraw[black] (-2,0) circle (2pt);
	\filldraw[black] (2,1) circle (2pt);
	\filldraw[black] (2,-1) circle (2pt);

	\node[above] at (-1,0) {$a$};
	\node[above] at (1,0.5) {$b$};
	\node[below] at (1,-0.5) {$c$};

	\node[left] at (-2,0) {$u$};
	\node[right] at (2,1) {$v$};
	\node[right] at (2,-1) {$w$};
\end{tikzpicture}
\]
Let $A = \{u,v,w\}$.
Then 
$$
D[A] = \begin{bmatrix}
0 & a + b & a + c \\
a + b & 0 & b + c \\
a + c & b + c & 0
\end{bmatrix}.
$$
and
\begin{align*}
\det D[A] &= 2(a+b)(a+c)(b+c) 
= 2\left( (a+b+c)(ab + ac + bc) - abc \right).
\end{align*}
\end{eg}


\section*{Acknowledgements}
The authors would like to thank ...


\bibliography{tree-distance-ref} 
\bibliographystyle{abbrv}

\end{document}
